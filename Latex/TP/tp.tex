\documentclass{article}
\usepackage[utf8]{inputenc}

\usepackage{amsmath, amssymb, amsthm}
\usepackage{graphicx, float}
\usepackage{tabularx}   % Paquete para hacer las tablas
\usepackage{array}      % Paquete para 
\usepackage{changepage} % Paquete para cambiar la ubicacion de palabras en el margen
\usepackage{relsize}
\usepackage[absolute,overlay]{textpos}
\usepackage{dsfont}
\usepackage{textcomp}
\usepackage{transparent}

\graphicspath{{images/}}

\begin{document}


% Logo del DC
\begin{textblock*}{16cm}(13cm, 0.3cm)
    \includegraphics[width=0.5\textwidth]{dc_logo.png}
\end{textblock*}

% Logo de UBA
\begin{textblock*}{17cm} (1cm, 17cm)
    {\transparent{0.2}
    \includegraphics[width=0.5\textwidth]{uba_logo.jpg}
    }
\end{textblock*}

\begin{textblock*}{40cm} (12cm, 22cm)
    {\raggedright
    \textbf{Facultad de Ciencias Exactas y Naturales}\\
    \textbf{Universidad de Buenos Aires}\\
    Ciudad Universitaria - (Pabellon I/Planta Baja)\\
    Intendente Guiraldes 2610 - C1428EGA\\
    Ciudad Autónoma de Buenos Aires - Rep. Argentina\\
    Tel/Fax: (++54 +11) 4576-3300\\
    http://www.exactas.uba.ar
    }
\end{textblock*}

\begin{adjustwidth}{-3.5cm}{0pt}
\begin{flushleft}

% Titulo + Subtitulo
{\raggedright
\textbf{\Large Titulo: Trabajo Practico 1} \\
\textit{\large Subtitulo} \\
}
\noindent\makebox[\linewidth]{\rule{\paperwidth}{0.3pt}}
% Linea horizontal


% Fecha alineada a la izquierda y materia a la derecha
\noindent\makebox[\textwidth][l]{6 de abril de 2025 \noindent\makebox[\textwidth][r]Algoritmos y Estructura de Datos}\\

% Nombre del grupo
\begin{textblock*}{15cm} (9.5cm, 7cm)
    \textbf{Grupo 30}
\end{textblock*}



% Cuadro de integrantes
\begin{textblock*}{14.5cm} (3.4cm, 8cm)
\begin{table}[h]
\centering
\begin{tabular}{|>{\vspace{1mm}\centering\arraybackslash}m{2.8cm} >{\vspace{2mm}\arraybackslash}m{1cm} >{\vspace{2mm}\arraybackslash}m{3cm}|}
    \hline
    Integrante & LU & Correo electrónico \\ [0.1cm]
    \hline
\end{tabular}
\begin{tabular}{|>{\vspace{1mm}\centering\arraybackslash}m{2.5cm} >{\vspace{1mm}\centering\arraybackslash}m{1cm} >{\vspace{2mm}\centering\arraybackslash}m{3.3cm}|}
    \hline
    Rankov, Jorge & 714/23 & jrankov@dc.uba.ar \\ [0.2ex]
    Falbo, Tiziana & nnn/nn & nnn@gmail.com \\ [0.3ex]
    Facundo & nnn/nn & nnn@gmail.com \\ [0.3ex]
    Bautista & nnn/nn & nnn@gmail.com \\ [0.3ex]
    \hline
\end{tabular}
\end{table}
\end{textblock*}

\end{flushleft}
\end{adjustwidth}

\thispagestyle{empty}

\newpage % Fin de Titulo, inicio de Formulas y Funciones
\setcounter{page}{1}

% Formulitas

\section{TAD}
\textbf{TAD \textdollar BerretaCoin} \{

    Aca iria toda la datita del TAD pero falta hacerlo xd \\
\}\\\\

\section{Funciones}
\subsection{Procesos y Predicados}

\textbf{proc MontosDeUsuarios} \{

    \textbf{Asegura}: $\forall$id $\in$ sinRepetidos(Usuarios(Cripto.blockchain)) $\rightarrow$ id $\in$ res

    $\longleftrightarrow$ (esMaximo(MontoUsuario(Cripto.blockchain, id)); Montos(Usuarios(Cripto.blockchain)))
\}\\\\\\
\textbf{pred esMaximo} (Monto: $\mathds{Z}$, Montos: Seq\textless$\mathds{Z}$\textgreater) \{

    ($\forall$i $\in$ Montos) $\rightarrow_L$ Monto $\geq$ i \\
\}\\\\\\
\textbf{proc montoMedio} (S: Seq\textless Seq\textless$\mathds{Z}$\texttimes$\mathds{Z}$\texttimes$\mathds{Z}$\texttimes$\mathds{Z}$\textgreater\textgreater): $\mathds{Z}$ \{\\

    \textbf{Requiere}: $\forall$ bloque $\in S$, $|$bloque$|$ \textgreater 0 \\

    \textbf{Asegura}: res = $\frac{\sum\limits_{j=0}^{|S|-1} \sum\limits_{i=1}^{|S_{[j]}|-1} S_{[j][i][3]}}{\sum\limits_{j=0}^{|S|-1} (|S_{[j]}|-1)}$ \\
\}\\\\

\newpage % Fin de Procesos y Predicados, inicio de Auxiliares

\subsection{Auxiliares}

\textbf{aux sinRepetidos}(S: Seq\textless$\mathds{Z}$\textgreater): Seq\textless$\mathds{Z}$\textgreater = \\

    $[S_{[0]}] + \mathlarger\sum\limits_{i=1}^{|s|-1} ifThenElse(S_{[i]} \in SubSeq(S, 0, i-1); \emptyset; [S_{[i]}])$ \\\\\\\\
\textbf{aux Usuarios}(S: Seq\textless Seq\textless$\mathds{Z}$\texttimes$\mathds{Z}$\texttimes$\mathds{Z}$\texttimes$\mathds{Z}$\textgreater\textgreater): Seq\textless$\mathds{Z}$\textgreater=

    \[
    \sum\limits_{i=0}^{|s|-i} \sum\limits_{j=0}^{|s_{[i]}|-1} \left(S_{[i][j][1]}, S_{[i][j][2]}\right)
    \]\\\\
\textbf{aux MontoUsuario} (S: Seq\textless Seq\textless $\mathds{Z}$\texttimes$\mathds{Z}$\texttimes$\mathds{Z}$\texttimes$\mathds{Z}$\textgreater\textgreater; id: $\mathds{Z}$): $\mathds{Z}$ =
    \[
    \sum\limits_{j=0}^{|s|-1} \left( \sum\limits_{i=0}^{|s_{[j]}|-1} 
    ifThenElse(id = S_{[j][i][1]}; S_{[j][i][3]}, 0) 
    - \sum\limits_{j=0}^{|s|-1} ifThenElse(id = S_{[j][i][2]}; S_{[j][i][3]}, 0) \right)
    \]\\\\
\textbf{aux Montos} (S: Seq\textless$\mathds{Z}$\textgreater): Seq\textless$\mathds{Z}$\textgreater = \\

    $\mathlarger\sum\limits_{j=0}^{|s|-1} (MontoUsuario(S_{[i]}))$



\end{document}

% Cosas a tener en cuenta:
% Requiere: id_comprador != id_vendedor
% Requiere: en id_transaccion, montoUsuario >= monto


% Funciones a implementar:

% agregarBloque

% maximosTenedores
% puedo usar MontosDeUsuarios para tomar a los maximos 
% Veo quien tiene la mayor cantidad de cripto, si se repite el maximo en mas usuarios,
% Los agrego a la lista. Sino, devuelvo una lista con 1 solo user

% Transaccion de creacion = la primer transaccion de cada bloque?
% montoMedio = for i in Montos, totalMontos += i -> res = totalMontos / |Montos|
